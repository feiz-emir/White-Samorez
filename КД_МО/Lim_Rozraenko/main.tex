\documentclass[a4paper,12pt]{article} % тип документа

%  Русский язык
\usepackage{multirow}
\usepackage{wrapfig}
\usepackage[T2A]{fontenc}			% кодировка
\usepackage[utf8]{inputenc}			% кодировка исходного текста
\usepackage[english,russian]{babel}	% локализация и переносы

\usepackage{indentfirst} %Красная строка
\usepackage[a4paper,top=1.3cm,bottom=2cm,left=1.5cm,right=1.5cm,marginparwidth=0.75cm]{geometry}
\usepackage[usenames]{color}
\usepackage{colortbl}
\usepackage{float}

\usepackage{graphicx}%картинки
\usepackage{textcomp}%Номер
\usepackage{wrapfig}%обтекание текстом теблиц и картинок
%гиперссылки
\usepackage{hyperref}
\usepackage[rgb]{xcolor}
\hypersetup{     %гипперсылки
 colorlinks=true, %false:ссылки в рамках
 urlcolor=blue   %на URL
 }
% Заметки
\usepackage{todonotes}
% Номера формул(необязятельна, см. по ситуации)
%\mathtoolsset{showonlyrefs=true} % Показывать номера только у тех формул, на которые есть \eqref{} в тексте.

% Математика
\usepackage{amsmath,amsfonts,amssymb,amsthm,mathtools} 
\usepackage{wasysym}

\usepackage{euscript} % Шрифт Евклид
\usepackage{mathrsfs} % Красивый матшрифт

\title{\textbf{Лабораторная работа. Определение константы диссоциации метилового оранжевого} }

\author{Алекснадров Максим, Кузнецов Роман, Тналиев Тимур \\ Б04-202}
\date{9 февраля 2024}

\begin{document}

\maketitle

\section{Введение:}
\paragraph{Цель работы:}
\quad
\\
\\ - Регистрация спектров поглощения растворов метилового оранжевоо с различными значениями pH в видимой и УФ-областях спектра;
\\ - Определение рабочих длин волн для кислой и основной форм исследуемого индикатора, нахождение изобестической точки;
\\ - Проверка закона Бугера-Ламберта-Бера; определение коэффициентов экстинкции кислой и основной форм индикатора на выбранных длинах волн;
\\ - Определение константы диссоциации метилового оранжевого.
\paragraph{Необходимое оборудование и материалы:}
\quad
\\
\\ -- Мерные колбы на 50 мл - 10 шт;
\\ -- Спектрофотометр, кварцевая кювета толщиной 1 см.
\\ -- Исходные готовые растворы (приготовлены лаборантом);
\\ \quad - Раствор метилового оранжевого 0.2 г/л;
\\ \quad - 0.1 М раствор HCL;
\\ \quad - 0.1 М раствор NaOH;
\\ -- Буферные растворы;
\\ -- Рабочие растворы.

\section{Ход работы:}
\paragraph{Определение коэффициентов экстинкции протонированной и депротонированной форм метилового оранжевого на длине на $\lambda_1$ = нм}
\paragraph{}
\begin{table}[H]
    \begin{center}
        \begin{tabular}{|c|c|}
        \hline
           Растворенное вещество & Концентрация\\\hline
           Метиловый оранжевый   & 1.0 г/л\\\hline
           HCl & 0.1 М \\\hline
           NaOH & 0.1 М\\\hline
           CH${_3}$COOH & 0.3 м \\\hline
        \end{tabular}
        \caption{Концентрации исходных растворов}
    \end{center}
\end{table}
\begin{table}[H]
    \begin{center}
        \begin{tabular}{|c|c|c|c|}
        \hline
            № р-ра & Р-р м-ж 0.2 г/л & Р-р К/Щ 0.1 н & Вода\\\hline
            1 & 2.0 мл & 5 мл HCl & Доб. воду в каждый р-р доводя его объем до 5 мл\\\hline
            2 & 1.5 мл & 5 мл HCl & Доб. воду в каждый р-р доводя его объем до 5 мл\\\hline
            3 & 1.0 мл & 5 мл HCl & Доб. воду в каждый р-р доводя его объем до 5 мл\\\hline
            4 & 0.5 мл & 5 мл HCl & Доб. воду в каждый р-р доводя его объем до 5 мл\\\hline
            5 & 2.5 мл & 5 мл NaOH & Доб. воду в каждый р-р доводя его объем до 5 мл\\\hline
            6 & 2.0 мл & 5 мл NaOH & Доб. воду в каждый р-р доводя его объем до 5 мл\\\hline
            7 & 1.5 мл & 5 мл NaOH & Доб. воду в каждый р-р доводя его объем до 5 мл\\\hline
            8 & 1.0 мл & 5 мл NaOH & Доб. воду в каждый р-р доводя его объем до 5 мл\\\hline
            9 & 2.0 мл & 25 мл буфер 1 & Доб. воду в каждый р-р доводя его объем до 5 мл\\\hline
            10 & 2.0 мл & 25 мл буфер 2 & Доб. воду в каждый р-р доводя его объем до 5 мл\\\hline
            11 & 2.0 мл & 25 мл буфер 3 & Доб. воду в каждый р-р доводя его объем до 5 мл\\\hline
            12 & 0 мл & 0 мл HCl & -\\\hline
        \end{tabular}
        \caption{Растворы}
    \end{center}
\end{table}
\quad
\\
\\Буферный раствор I (pH = 3.9) готовят в мерной колбе на 200 мл из ацетата натрия и уксусной кислоты. Может использоваться безводный ацетат натрия, дигидриат или тригидрат. Рассчитывают навеску ацетата натрия, чтобы его концентрация в готовом буферном растворе была равна 0.02 М. Необходимое количество уксусной кислоты рассчитывают по формуле Гендерсона.
\begin{equation}
    pH = - \log{K_{\alpha}(AcOH)} + \log{\frac{C(AcONa)}{C(AcOH)}}
\end{equation}
\\Буфер II (pH = 3.7) и буфер III(pH = 3.5) готовят из буфера I в стаканчиках на 50 мл, добавляя по каплям уксусную кислоту. Значение pH контролирует по ионометру.
\\Раствор метилового оранжевого с концентрацией 0,2 г/л готовят в мерной колбе 50 мл.
\\Из этих растворов готовят рабочие растворы в мерных колбах на 50 мл (доводят до метки водой).

\quad
\\
\\Были выбраны следующие рабочие длины волны:
\\1) 470 нм;
\\2) 510 нм;
\paragraph{}
\\
Находим угловые коэффициенты для графиков 4-7. Из них получаем коэффициенты экстинкции двух форм индикатора на двух выбранных значениях рабочей длины волны:
\\1) $\varepsilon_{\gamma}$ = 5333 л/(моль*см) (4 график)
\\2) $\varepsilon_{\gamma}$ = 6190 л/(моль*см) (5 график)
\\3) $\varepsilon_{\gamma}$ = 3200 л/(моль*см) (6 график)
\\4) $\varepsilon_{\gamma}$ = 1538 л/(моль*см) (7 график)

\paragraph{}
\\
Посчитаем $D_{kisl}$ и $D_{shel}$ по формулам:
\begin{equation}
    D_{kisl} = \varepsilon_{HA}l C_0
\end{equation}

\begin{equation}
    D_{shel} = \varepsilon_{A_{-}}l C_0
\end{equation}

\quad
\\
\\1) $D_{kisl}$ ($\lambda_1$) =  0.152
\\2) $D_{kisl}$ ($\lambda_2$) =  0.131
\\3) $D_{shel}$ ($\lambda_1$) =  0.037
\\4) $D_{shel}$ ($\lambda_2$) = 0.078

\paragraph{}
\quad
\\
\\Посчитаем степени диссоциации:
\\1) $\lambda_1 (9,10,11)$: $\alpha_1$ = 0.602, 0.751, 0.077.
\\2) $\lambda_2 (9,10,11)$: $\alpha_2$ = 1.07, 1.31, 1.05.

\newpage
\section{Приложения:}

\begin{figure}[!h]
\begin{center}
\includegraphics[width=1.1\linewidth]{graph_1_new.png}
    \begin{center}
    \caption{Спектры всех растворов}
    \end{center}
\end{center}
\end{figure}

\begin{figure}[!h]
\begin{center}
\includegraphics[width=1.0\linewidth]{graph_2_new_new.png}
    \begin{center}
    \caption{Спектры всех растворов кроме воды}
    \end{center}
\end{center}
\end{figure}

\newpage

\begin{figure}[!h]
\begin{center}
\includegraphics[width=1.1\linewidth]{graph3_1691011.png}
    \begin{center}
    \caption{Спектры растворов 1,6,9,10,11}
    \end{center}
\end{center}
\end{figure}

\begin{figure}[!h]
\begin{center}
\includegraphics[width=0.75\linewidth]{graph4.jpg}
    \begin{center}
    \caption{График зависимости D от С(HCl($\lambda_2$)}
    \end{center}
\end{center}
\end{figure}

\newpage

\begin{figure}[!h]
\begin{center}
\includegraphics[width=0.75\linewidth]{graph5.jpg}
    \begin{center}
    \caption{График зависимости D от С(HCl($\lambda_1$)}
    \end{center}
\end{center}
\end{figure}

\begin{figure}[!h]
\begin{center}
\includegraphics[width=0.75\linewidth]{graph6.jpg}
    \begin{center}
    \caption{График зависимости D от С(NaOH($\lambda_2$)}
    \end{center}
\end{center}
\end{figure}


\newpage

\begin{figure}[!h]
\begin{center}
\includegraphics[width=0.75\linewidth]{graph7.jpg}
    \begin{center}
    \caption{График зависимости D от С(NaOH($\lambda_1$)}
    \end{center}
\end{center}
\end{figure}
\end{document}
