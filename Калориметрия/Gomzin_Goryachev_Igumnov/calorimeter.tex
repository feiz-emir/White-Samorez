\documentclass[a4paper]{article}
\usepackage[utf8]{inputenc}
\usepackage[russian]{babel}
\usepackage[T2]{fontenc}
\usepackage[warn]{mathtext}
\usepackage{graphicx}
\usepackage{amsmath}
\usepackage{chemarrow}
\usepackage{floatflt}
\usepackage[left=20mm, top=20mm, right=20mm, bottom=20mm, footskip=10mm]{geometry}




\newpage
\begin{document}

\begin{titlepage}
	\centering
	\vspace{5cm}
	{\scshape\LARGE Московский физико-технический институт \par}
	\vspace{4cm}
	{\scshape\Large Отчет по лабораторной работе \par}
	\vspace{1cm}
	{\huge\bfseries Определение теплоты растворения неизвестной соли \par}
	\vspace{1cm}
	\vfill
\begin{flushright}
	{\large выполнили студенты группы Б04-202}\par
	\vspace{0.3cm}
	{\LARGE Гомзин Александр} \par
		\vspace{0.3cm}
	{\LARGE Горячев Арсений} \par
        \vspace{0.3cm}
        {\LARGE Игумнов Дмитрий} \par
\end{flushright}
	

	\vfill

	Долгопрудный, 2024 г.
\end{titlepage}

	\thispagestyle{empty}


	\newpage \LARGE
	
		\tableofcontents % Вывод содержания
	
	\newpage
\par
	\large \textbf{\sffamily{Цель работы:}} определить сумарную теплоемкость системы (постоянную калориметрической системы); определить интегральную теплоту растворения неизвестной соли.
	\par \vspace{0.3 cm}
	\textbf{\sffsamily{В работе используются:}}
        \begin{itemize}
           \item калориметр
           \item пластиковый стакан на 250 мл
           \item мерный цилиндр
           \item мешалка
           \item термометр
           \item стакан с точно взвешенной навеской известной соли (KCl)
           \item стакан с точно взвешенной навеской неизвестной соли
           \item дистиллированная вода
        \end{itemize}  
\vspace{0.3 cm}
	\section{\LARGE \textbf{Теоретические сведения}}
Интегральная теплота растворения – тепловой эффект, сопровождающий растворение 1 грамма (удельная) или 1 моля (молярная) твердого вещества в воде. \par \vspace{0.3 cm}
Для нахождения интегральной теплоты растворения воспользуемся методом 
калориметрии: будем фиксировать изменение температуры
калориметре при растворении в ней известной соли (KCl) в разных количествах, таким образом определим 
суммарную теплоемкость калориметрической системы:

\begin{center}
\begin{mathmode}

\LARGE K = Q/\Delta T - c_wm_w

\par \vspace{0.3 cm}

Q = --\Delta H_m \cdot \frac{m(KCl)}{\mu(KCl)}


\end{mathmode}

\par \vspace{0.3 cm}
\end{center}

Окончательно выражаем интегральную теплоту растворения следующим образом:

\begin{center}
\begin{mathmode}

\LARGE \Delta H_m = - Q_x \cdot \frac{\mu_X}{m_X}

\par \vspace{0.3 cm}
\end{mathmode}

\end{center}











\par \vspace{1cm}

\section{\LARGE Ход Работы}

На основании методички. \par \vspace{0.3cm}

\subsection{\Large Существенные замечания}

В ходы опыта в силу гигроскопичности неизвестной соли возникла проблема с закупориванием воронки при засыпании данной соли в калориметр. В результате было принято решение о засыпании при снятой крышке калориметра, что приводило к падению показаний термометра на 2-3 сотых градуса, но она быстро восстанавливалась.




\section{\LARGE \textbf{Экспериментальные данные}}
\vspace{1.5 cm}

\par

\begin{center}
\textbf{Таблица 1.} Массы навесок известной соли.

\vspace{0.3cm}
\begin{tabular}{|l|c|r|c|c|c|c|c|c|}
    \hline
    \textnumero & 1 & 2 & 3 & 4 & 5\\
    \hline
    m(KCl), г & 2.002 & 4.000 & 6.001 & 8.018 & 10.014 \\
    \hline
     
\end{tabular}
\par \vspace{1 cm}

Масса неизвестной соли: \begin{mathmode} m_X = 4.006 \end{mathmode}

\end{center}
\par \vspace{1 cm}


\section{\LARGE \textbf{Обработка результатов}}
\par

\graphicspath{{./images/}}
		\begin{center}
		
			\includegraphics[scale=1.1]{cal1.png}
	

	\par
 \vspace{0.3cm}
 \end{center}


\graphicspath{{./images/}}
		\begin{center}
		
			\includegraphics[scale=1.1]{cal2.png}
	

	\par
 \vspace{0.3cm}
 \end{center}
\vspace{0.3cm}
\par


\graphicspath{{./images/}}
		\begin{center}
		
			\includegraphics[scale=1.1]{cal3.png}
	

	\par
 \vspace{0.3cm}
 \end{center}
\vspace{0.3cm}
\par

\graphicspath{{./images/}}
		\begin{center}
		
			\includegraphics[scale=1.1]{cal4.png}
	

	\par
 \vspace{0.3cm}
 \end{center}
\vspace{0.3cm}
\par

\graphicspath{{./images/}}
		\begin{center}
		
			\includegraphics[scale=1.1]{cal5.png}
	

	\par
 \vspace{0.3cm}
 \end{center}
\vspace{0.3cm}
\par

Калориметрическая постоянная (усреднение по 5 измерениям):
\begin{center}
\begin{mathmode}

\LARGE K = (946 \pm 33) \end{mathmode} \LARGE Дж/К
\end{center}

\graphicspath{{./images/}}
		\begin{center}
		
			\includegraphics[scale=1.1]{cal6.png}
	

	\par
 \vspace{0.3cm}
 \end{center}
\vspace{0.3cm}
\par

\newpage
Визуализация расчетов для определения соли (выбор солей на основании данных справочника из лаборатории):

\graphicspath{{./images/}}
		\begin{center}
		
			\includegraphics[scale=1.1]{cal7.png}
	

	\par
 \vspace{0.3cm}
 \end{center}
\vspace{0.3cm}
\par


\begin{center}
\begin{mathmode}

\LARGE \Delta H_{solv}(X) = (303 \pm 42) \end{mathmode} \LARGE кДж/г
\end{center}
\par \vspace{0.3 cm}

Видим по графику, что наименьшее отклонение от табличных значений получается, если предположить, что соль X -- хлорид или нитрат аммония. Приведем относительное отклонение от табличных данных для них:

\begin{center}
\begin{mathmode}

\LARGE \varepsilon[NH_4Cl] \approx 0.06, \hspace{0.3 cm} \varepsilon[NH_4NO_3] \approx 0.05

\end{mathmode}
\end{center}
\par \vspace{0.3 cm}



\section{\LARGE \textbf{Заключение}}
\par \vspace{0.3 cm}

В ходе работы нами были получены величины постоянной калориметра и интегральной теплоты растворения соли X в расчете на единицу массы (удельную). На основании данных справочника из лаборатории также были проведены расчеты, позволяющие предположить возможную соль.
\par \vspace{0.3 cm}
По результатам вычислений ближе всего к табличным молярным энтальпиям получаются энтальпии в предположении о том, что неизвестная соль -- хлорид аммония или нитрат аммония, и на основании качественных опытов с концентрированной серной кислотой и нитратом серебра было получено, что среди табличных солей подходящим является только \begin{mathmode} NH_4Cl \end{mathmode}.
yep.



\end{document}	


