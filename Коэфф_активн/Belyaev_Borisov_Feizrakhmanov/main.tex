\documentclass[a4paper,12pt]{article} % добавить leqno в [] для нумерации слева
\usepackage[a4paper,top=1.3cm,bottom=2cm,left=1.5cm,right=1.5cm,marginparwidth=0.75cm]{geometry}
%%% Работа с русским языком
\usepackage{cmap}					% поиск в PDF
\usepackage[warn]{mathtext} 		% русские буквы в фомулах
\usepackage[T2A]{fontenc}			% кодировка
\usepackage[utf8]{inputenc}			% кодировка исходного текста
\usepackage[english,russian]{babel}	% локализация и переносы
\usepackage{physics}
\usepackage{multirow}
\usepackage{pgfplots}
\pgfplotsset{compat=1.9}
\usepackage{caption}
\usepackage{float}




%%% Нормальное размещение таблиц (писать [H] в окружении таблицы)
\usepackage{float}
\restylefloat{table}


\usepackage{graphicx}

\usepackage{wrapfig}
\usepackage{tabularx}


%%% Дополнительная работа с математикой
\usepackage{amsmath,amsfonts,amssymb,amsthm,mathtools} % AMS
\usepackage{icomma} % "Умная" запятая: $0,2$ --- число, $0, 2$ --- перечисление

%% Номера формул
%\mathtoolsset{showonlyrefs=true} % Показывать номера только у тех формул, на которые есть \eqref{} в тексте.

%% Шрифты
\usepackage{euscript}	 % Шрифт Евклид
\usepackage{mathrsfs} % Красивый матшрифт
\usepackage{pgfplots}
\pgfplotsset{compat=1.9}

%% Свои команды
\DeclareMathOperator{\sgn}{\mathop{sgn}}

%% Перенос знаков в формулах (по Львовскому)
\newcommand*{\hm}[1]{#1\nobreak\discretionary{}
	{\hbox{$\mathsurround=0pt #1$}}{}}

\date{\today}

\begin{document}

\begin{titlepage}
	\begin{center}
		{\large МОСКОВСКИЙ ФИЗИКО-ТЕХНИЧЕСКИЙ ИНСТИТУТ (НАЦИОНАЛЬНЫЙ ИССЛЕДОВАТЕЛЬСКИЙ УНИВЕРСИТЕТ)}
	\end{center}
	\begin{center}
		{\large Физтех-школа электроники, фотоники и молекулярной физики}
	\end{center}
	
	
	\vspace{4.5cm}
	{\huge
		\begin{center}
			{Лабораторная работа по химической физике}\\
                {"Изучение термодинамических параметров гетерогенных
реакций ионного обмена методом потенциометрии".}\\
		\end{center}
	}
	\vspace{2cm}
	\begin{flushright}
		{\LARGE Авторы:\\ Беляев Юрий \\ Борисов Павел  \\
		Фейзрахманов Эмир \\	\vspace{0.2cm}
			группа Б04-202}
	\end{flushright}
	\vspace{7cm}
	\begin{center}
		\today
	\end{center}
\end{titlepage}




\paragraph{Цель работы:} 
 определение изотерм ионного обмена между водными растворами электролитов
и ионитами (катионитами на основе полисурьмяной кислоты) методом потенциометрии.

\paragraph{Оборудование:}
\begin{enumerate}
    \item Иономер И-160М
  \item Магнитная мешалка
 \item Аналитические весы
  \item Микрошпатель для порошков
  \item Мерный цилиндр объемом 100 мл
 \item  Стеклянный рН - чувствительный электрод
  \item Хлорсеребряный электрод сравнения
  \item Дозатор переменного объема 20-100 мкл
  \item Стеклянный стакан объемом 100 мл (3 шт.)
  \item Катионит – полисурьмяная кислота (порошок)
  \item Раствор хлорида лития (1 М)
  \item Растворы хлорида и гидроксида натрия (1 М)
  \item Раствор хлорида калия (1 М)
  \item Деионизованная вода
\end{enumerate}

\section{Теоретические сведения}
\textit{Иониты (ионообменники)} - твёрдые нерастворимые вещества, содержащие кислотные (катиониты) или основные (аниониты) группировки, способные обменивать свои ионы (катионы или анионы) на ионы контактирующего с ними раствора. В нашей работе в качестве ионита мы будем использовть катионит - полисурьмяную кислоту.

Как мы уже сказали, можно провести такую классификацию ионитов:

\begin{itemize}
    \item Катиониты. Являясь кислотами, они поглощают положительные ионы и обменивают их на другие положительные ионы.
    \item Аниониты. Являясь основаниями, поглощают отрицательные ионы и обменивают их на другие отрицательные ионы. Для регенерации анионита его подвергают действию щёлочи.
    \item Амфотерные иониты или полиамфолиты. В разных ситуациях они могут вести себя или как катиониты, или как аниониты. Для регенерации амфотерных ионитов их промывают водой.
\end{itemize}

Принцип работы ионита такой: рассмотрим для определённости катионит, содержащий катионы водорода. Если через такой катионит пропустить вещество без ионов, например, дистиллированную воду, то ни вещество, ни катионит никак не изменятся. Однако если пропустить раствор соли, то этот раствор превратится в кислоту, а катионит будет содержать уже не катионы водорода, а катионы соли — произойдёт ионообмен. Чтобы вернуть катионит в исходное состояние, через него нужно пропустить кислоту — катионы соли в катионите вновь заменятся на катионы водорода — а затем отмыть от остатков кислоты. Подобным же образом аниониты обменивают свои анионы на анионы среды, в которую их помещают.

Определяя обмен ионов как гетерогенную реакцию двойного обмена двух электролитов, один из
которых является ионитом, было получено уравнение изотермы обмена в общем виде для обоих механизмов и их
сочетания. Его решение 

\begin{equation}
    \lg\textit{a}_{{H}^{+}} = (z_{{H}^{+}} / \textit{z}_{{Me}^{+}}) \lg\textit{a}_{{Me}^{+}} + f(\textit{Г}_{{Me}^{+}}),  \textrm{ где} 
\end{equation}

\hspace{1 cm}$z$ - стехиометрический коэффициент,

\hspace{1 cm}$a$ - активность ионов в растворе, 

\hspace{1 cm}$\textit{Г}$ - обменная ёмкость,

\hspace{1 cm}$f(\textit{Г})$ - некоторая функция от $\textit{Г}$.

Конкретный вид функции был найден только для конкретного случая - монофункционального ионита (имеющего всего один тип активных групп, т.е. в ионите одинаковые функциональные группы) в предположении равноценности всех связей между каждым из двух обменивающихся ионов и независимости взаимодействия ионов от степени заселенности твердой фазы. В этом случае уравнение изотермы обмена
принимает форму закона действующих масс:

\begin{equation}
    K_a = \frac{\textit{Г}_{{Me}^{+}}}{\textit{Г}_{{H}^{+}}} \cdot \frac{\textit{a}_{{H}^{+}}}{\textit{a}_{{Me}^{+}}},  \textrm{ где}
\end{equation}

\hspace{1 cm}$K_a$ – константа кислотности.

\textit{Активность} ионов $a$ в растворе какого-либо вещества связана с концентрацией этих ионов $C$ через безразмерный коэффициент активности $\gamma$:

\begin{equation}
    a = \gamma C.
\end{equation}
Для сильно разбавленных растворов $\gamma \approx 1$. В нашем эксперименте считаем, что $\gamma = 1$.

Зная показатель $pH$ раствора, можно определить концентрацию ионов водорода $С_{H^{+}} = [H^{+}]$ в нем в соответствии с формулой
\begin{equation}
    pH = - lg [H^{+}].
    \label{eq:eq1}
\end{equation}

Количественной характеристикой ионита является \textit{предельная обменная емкость} $\textit{Г}_0$ - количество вещества (в молях), поглощенного до полного насыщения ионита в равновесных условиях на единицу массы ионита. Полную обменную емкость ионита определяет количество способных к обмену противоионов.

В нашем эксперименте будем считать её как

\begin{equation}
    \textit{Г}_0 = \frac{\nu_0}{m} = \frac{\Delta \nu_1 + \Delta \nu_2 + \Delta \nu_3}{m}, \textrm{ где}
\end{equation}

\hspace{1 cm}$\Delta \nu_1$ -  количество ионов калия, замещающих протоны в катионите после его добавления в воду (определяется по начальному изменению pH в течение первых 5 минут после добавления в дистиллированную воду катионита),

\hspace{1 cm}$\Delta \nu_2$ - максимальное количество ионов металла, замещающих протоны в результате приливаний раствора хлорида металла в течение длительного времени,

\hspace{1 cm}$\Delta \nu_3$ - количество ионов натрия, замещающих протоны в результате
титрования щелочью,

\hspace{1 cm}$m$ - масса ионита.

Так же вводятся понятия \textit{обменных ёмкостей} 
$\textit{Г}_{{Me}^{+}}$ и $\textit{Г}_{{H}^{+}}$ - количества веществ, поглощенных при равновесии в данных рабочих условиях:
\begin{equation}
    \textit{Г}_{{Me}^{+}} = \frac{\nu_{{Me}^+}^{(cat.)}}{m}, \quad \textit{Г}_{{H}^{+}} = \frac{\nu_{{H}^+}^{(cat.)}}{m} = \frac{\nu_0 - \nu_{{Me}^+}^{(cat.)}}{m} = \textit{Г}_0 - \textit{Г}_{{Me}^{+}}.
\end{equation}

Отсюда
\begin{equation}
    \textit{Г}_{{Me}^{+}} + \textit{Г}_{{H}^{+}} = \textit{Г}_0 = const.
\end{equation}


Равновесное количество ионов металла $\nu_{Me}^{(cat.)}$ в катионите можно выразить как функцию номера прилитой порции раствора хлорида метала (носителя свободных ионов, с которыми и будет обмениваться катионит своими протонами):

\begin{equation}
    \nu_{{Me}^+}^{(cat.)} [1] = [H^{+}]_1 \cdot (V_0 + V_1) -[H^{+}]_0 \cdot V_0, \textrm{ где}
\end{equation}

\hspace{1 cm}$[H^{+}]_0$ - начальная концетрация ионов $H^{+}$ в растворе после добавления катионита,

\hspace{1 cm}$V_0$ - начальный объем всего раствора катионита в воде,

\hspace{1 cm}$[H^{+}]_1$ - концетрация протонов в растворе после добавления одной порции раствора хлорида металла,

\hspace{1 cm}$V_1$ - объем добавленной порции раствора катионита металла.

Тогда воспользовавшись формулой \eqref{eq:eq1} получаем
\begin{equation}
  \nu_{{Me}^+}^{(cat.)} [1] = -10^{-pH_0} \cdot V_0 + 10^{-pH_1} \cdot (V_0 + V_1).
\end{equation}
Аналогично концетрация после $k$-ой порции раствора хлорида металла
\begin{equation}
    \nu_{{Me}^+}^{(cat.)} [k] =\nu_{{Me}^+}^{(cat.)}[k-1] -10^{-pH_{k-1}} \cdot (V_0 + \sum_{i=1}^{k-1}V_i) + 10^{-pH_k} \cdot (V_0 + \sum_{i=1}^{k}V_i).
\end{equation}

Равновесное количество ионов металла $\nu_{Me}^{(solv.)}$ в растворе можно выразить как функцию номера прилитой порции раствора хлорида:

\begin{equation}
    \nu_{{Me}^+}^{(solv.)} [1] = C_{MeCl} \cdot V_1 - \nu_{{Me}^+}^{(cat.)} [1] =  C_{MeCl} \cdot V_1 + [H^{+}]_0 \cdot V_0 - [H^{+}]_1 \cdot (V_0 + V_1) =
\end{equation}


\begin{equation}
      = C_{MeCl} \cdot V_1 + 10^{-pH_0} \cdot V_0 - 10^{-pH_1} \cdot (V_0 + V_1),
\end{equation}

\begin{equation}
    \nu_{{Me}^+}^{(cat.)} [k] = \nu_{{Me}^+}^{(solv.)} [k-1] + C_{MeCl} \cdot V_k + 10^{-pH_{k-1}} \cdot (V_0 + \sum_{i=1}^{k-1}V_i) - 10^{-pH_k} \cdot (V_0 + \sum_{i=1}^{k}V_i).
\end{equation}

В нашей работе мы определим предельную обменную ёмкость, построим изотермы ионного обмена и определим константу кислотности. Фиксировать мы будем количество добавляемых веществ в раствор с помощью пипетки, которой мы будем добавлять эти вещества, и изменение кислотности раствора по показанию pH-метра, откуда сможем найти обменные емкости и концентрации.

\begin{figure}[H]
    \centering
    \includegraphics[width=10cm]{фото3.jpg}
    \label{fig:enter-label}
    \caption{Схема работы индикаторных электродов.}
\end{figure}
Дли измерения pH мы будем использовать \textit{стеклянный электрод}. 

\begin{figure}[H]
    \centering
    \includegraphics[width=16cm]{Снимок экрана (84).png}
    \label{fig:enter-label}
\end{figure}

Для определения рН в исследуемый раствор погружается стеклянный индикаторный электрод и
проточный хлорсеребряный электрод сравнения. В комбинированном стеклянном электроде в
одну трубку помещены хлорсеребряный электрод сравнения и стеклянный с разделением
электролитических контактов и контактов к прибору. В обоих случаях измерение рН со
стеклянным электродом сводится к измерению ЭДС цепи:

\begin{equation}
    \textrm{ЭДС} = E_1 + E_2 +E_3 +E_4.
\end{equation}

\begin{figure}[H]
    \centering
    \includegraphics[width=10cm]{Снимок экрана (85).png}
    \label{fig:enter-label}
    \caption{Схема электродов, используемых в работе.}
\end{figure}

Очевидно, что переменной величиной, зависящей от рН исследуемого раствора, является только
E1 и, соответственно, величина измеряемой ЭДС

\begin{equation}
    \textrm{ЭДС} = const - 2,303 \frac{RT}{F} \ln a_{{H}^+} = const - 2,303 \frac{RT}{F} \ln ph
\end{equation}

В константу, обозначаему $E^0_{\textrm{стекла}}$ входят потенциалы внешнего и
внутреннего электродов сравнения, а также потенциал асимметрии.

\section{Ход работы}

\subsection{Эксперимент с хлоридом натрия.}

Кислотность воды перед началом эксперимента:

\begin{equation*}
    pH (H_20)= 7,06.
\end{equation*}

Аккуратно засыпем в воду 0,25 г. катионита и будем фиксировать значения рН в течение первых 5 минут после добавления.

\begin{table}[H]
    \centering
    \begin{tabular}{|l|l|l|l|l|l|l|l|l|l|l|l|}
    \hline
        t, мин & 0,5 & 1 & 1,5 & 2 & 2,5 & 3 & 3,5 & 4 & 4,5 & 5 & 8 \\ \hline
        pH & 4,2 & 3,92 & 3,84 & 3,81 & 3,79 & 3,76 & 3,76 & 3,76 &  3,75 & 3,76 & 3,78 \\ \hline
    \end{tabular}
    \caption{Значения pH раствора после добавления катионита.}
\end{table}

В результате мы получили белую суспензию растворенного в воде катионита. Установившаяся кислотность равна 
 \begin{equation*}
     pH(\textit{суспензии}) = 3,78.
 \end{equation*}



Будем добавлять в неё по 50 мкл. 1 М раствора NaCl,
каждый раз фиксируя равновесное значение рН, до тех пор, пока не установится постоянное значение pH. Затем проведём титрование полученной суспензии 1 М раствором NaOH и построим кривую титрования.

\begin{table}[H]
    \centering
    \begin{tabular}{|l|l|l|l|l|l|}
    \hline
        V(р-ра), мл & V(NaCl), мкл & pH & $a_{H^{+}}$, ммоль/л & $a_{Na^{+}}$, ммоль/л & $a_{Na^{+}}$/$a_{H^{+}}$ \\ \hline
        100 & 0 & 3,69 & 0,2  & 0,046 & 0,22\\ \hline
        100,05 & 50 & 3,3 &  0,5 & 0,66 & 1,32\\ \hline
        100,1 & 100 & 3,12 & 0,75 & 0,9 & 1,19 \\ \hline
        100,15 & 150 & 3,04 & 0,91 & 1,24 & 1,37\\ \hline
        100,2 & 200 & 2,99 & 1,02 & 1,63 & 1,59\\ \hline
        100,25 & 250 & 2,98 & 1,05  & 2,11 & 2,01\\ \hline
        100,3 & 300 & 2,97 & 1,07  &  2,58 & 2,4\\ \hline
        100,35 & 350 & 2,96 & 1,09  &  3,05 &2,78 \\ \hline
        100,4 & 400 & 2,96 &  1,1 & 3,55 & 3,23\\ \hline
    \end{tabular}
    \caption{Уменьшение pH суспензии при добавлении NaCl.}
\end{table}

\begin{table}[H]
\centering
    \begin{tabular}{|l|l|l|}
    \hline
        V(р-ра), мл & V(NaOH), мкл & pH \\ \hline
        100,4 & 0 & 3,02 \\ \hline
        100,45 & 50 & 3,04 \\ \hline
        100,5 & 100 & 3,07 \\ \hline
        100,55 & 150 & 3,11 \\ \hline
        100,6 & 200 & 3,15 \\ \hline
        100,65 & 250 & 3,23 \\ \hline
        100,7 & 300 & 3,3 \\ \hline
        100,75 & 350 & 3,4 \\ \hline
        100,8 & 400 & 3,53 \\ \hline
        100,85 & 450 & 3,73 \\ \hline
        100,9 & 500 & 3,96 \\ \hline
        100,95 & 550 & 4,31 \\ \hline
        101 & 600 & 4,75 \\ \hline
        101,05 & 650 & 5,3 \\ \hline
        101,1 & 700 & 5,92 \\ \hline
        101,15 & 750 & 6,57 \\ \hline
        101,2 & 800 & 7,72 \\ \hline
        101,25 & 850 & 8,63 \\ \hline
    \end{tabular}
    \caption{Увеличение pH суспензии при титровании NaOH.}
\end{table}

\begin{figure}[H]
    \centering
    \caption{Кривая титрования суспензии раствором гидрооксида натрия.}
\begin{tikzpicture}
\begin{axis} [
    xlabel = {V, \textit{мкл}},
    ylabel = {pH},
    width = 15 cm,
    height = 15 cm,
    grid = major,
    ymin = 0,
    xmin = 0,
    xmax = 900,
    ymax = 10
]
\addplot table [x = a, y = b]{
	a     b    
	0     3.02  
        50    3.04  
        100   3.07 
        150   3.11 
        200   3.15
        250   3.23 
        300   3.3 
        350   3.4 
        400   3.53 
        450   3.73 
        500   3.96 
        550   4.31 
        600   4.75 
        650   5.3 
        700   5.92 
        750   6.57 
        800   7.72 
        850   8.63    
};
\end{axis}
\end{tikzpicture}
\end{figure}

По кривой титрования определим ёмкость катионита в единицах мг-экв/г:

\begin{equation*}
      \textit{Г}_0^{(1)} = \frac{(10^{-3,78}-10^{-7,06}) \cdot 0,1 + (10^{-2,96} \cdot 0,104 - 10^{-3.78}\cdot 0,1 ) + (10^{-7,06} \cdot 0,112 - 10^{-2,96} \cdot 0,104)} {m} = 
\end{equation*}

\begin{equation*}
     = \frac{10^{-7,06} \cdot 0,112 - 10^{-7,06} \cdot 0,1 }{m} = 0,405 \; \frac{ммоль.}{г.}
\end{equation*} 

\subsection{Эксперимент с хлоридом калия.}

Кислотность воды перед началом эксперимента:

\begin{equation*}
    pH (H_20)= 7,55.
\end{equation*}

Аккуратно засыпем в воду 0,25 г. катионита и будем фиксировать значения рН в течение первых 5 минут после добавления.

\begin{table}[H]
    \centering
    \begin{tabular}{|l|l|l|l|l|l|l|}
    \hline
        t, мин & 0,5 & 1 & 1,5 & 2 & 2,5 & 3 \\ \hline
        pH & 4,47 & 4,38 & 4,32 & 4,33 & 4,34 & 4,33 \\ \hline
    \end{tabular}
     \caption{Значения pH раствора после добавления катионита.}
	\label{tab:my-table1}
\end{table}

В резултате мы получили белую суспензию растворенного в воде катионита. Установившаяся кислотность равна 
 
 \begin{equation*}
     pH(\textit{суспензии}) = 4,33.
 \end{equation*}


Будем добавлять в неё по 50 мкл. 1 М раствора KCl,
каждый раз фиксируя равновесное значение рН, до тех пор, пока не установится постоянное значение pH.

\begin{table}[H]
    \centering
    \begin{tabular}{|l|l|l|}
    \hline
        V(р-ра), мл & V(KCl), мкл & pH \\ \hline
        100 & 0 & 4,41 \\ \hline
        100,05 & 50 & 4,07 \\ \hline
        100,1 & 100 & 3,88 \\ \hline
        100,15 & 150 & 3,78 \\ \hline
        100,2 & 200 & 3,69 \\ \hline
        100,25 & 250 & 3,65 \\ \hline
        100,3 & 300 & 3,6 \\ \hline
        100,35 & 350 & 3,58 \\ \hline
        100,4 & 400 & 3,56 \\ \hline
        100,45 & 450 & 3,55 \\ \hline
    \end{tabular}
    \caption{Уменьшение pH суспензии при добавлении KCL.}
\end{table}

\subsection{Эксперимент с хлоридом лития.}

Кислотность воды перед началом эксперимента:

\begin{equation*}
    pH (H_20)= 7,45.
\end{equation*}

Аккуратно засыпем в воду 0,25 г. катионита и будем фиксировать значения рН в течение первых 5 минут после добавления.

\begin{table}[H]
    \centering
     \begin{tabular}{|l|l|l|l|l|l|l|}
    \hline
        t, мин & 0,5 & 1 & 1,5 & 2 & 2,5 & 3 \\ \hline
        pH & 4,56 & 4,4 & 4,41 & 4,36 & 4,35 & 4,35 \\ \hline
    \end{tabular}
     \caption{Значения pH раствора после добавления катионита.}
\end{table}

В резултате мы получили белую суспензию растворенного в воде катионита. Установившаяся кислотность равна 
 \begin{equation*}
     pH(\textit{суспензии}) = 4,35.
 \end{equation*}


Будем добавлять в неё по 50 мкл. 1 М раствора LiCl,
каждый раз фиксируя равновесное значение рН, до тех пор, пока не установится постоянное значение pH. Затем проведём титрование полученной суспензии 1 М раствором NaOH и построим кривую титрования.

\begin{table}[H]
    \centering
    \begin{tabular}{|l|l|l|}
    \hline
        V(р-ра), мл & V(LiCl), мкл & pH \\ \hline
        100 & 0 & 4,31 \\ \hline
        100,05 & 50 & 3,96 \\ \hline
        100,1 & 100 & 3,84 \\ \hline
        100,15 & 150 & 3,77 \\ \hline
        100,2 & 200 & 3,73 \\ \hline
        100,25 & 250 & 3,68 \\ \hline
        100,3 & 300 & 3,66 \\ \hline
        100,35 & 350 & 3,64 \\ \hline
        100,4 & 400 & 3,61 \\ \hline
        100,45 & 450 & 3,59 \\ \hline
        100,5 & 500 & 3,57 \\ \hline
        100,55 & 550 & 3,55 \\ \hline
        100,6 & 600 & 3,51 \\ \hline
    \end{tabular}
    \caption{Уменьшение pH суспензии при добавлении LiCl.}
\end{table}

\begin{table}[H]
\centering
    \begin{tabular}{|l|l|l|}
    \hline
        V(р-ра), мл & V(NaOH), мкл & pH \\ \hline
        100,6 & 0 & 3,51 \\ \hline
        100,65 & 50 & 3,53 \\ \hline
        100,7 & 100 & 3,61 \\ \hline
        100,75 & 150 & 3,68 \\ \hline
        100,8 & 200 & 3,78 \\ \hline
        100,85 & 250 & 3,91 \\ \hline
        100,9 & 300 & 4,06 \\ \hline
        100,95 & 350 & 4,22 \\ \hline
        101 & 400 & 4,44 \\ \hline
        101,05 & 450 & 4,77 \\ \hline
        101,1 & 500 & 5,13 \\ \hline
        101,15 & 550 & 5,56 \\ \hline
        101,2 & 600 & 5,83 \\ \hline
        101,25 & 650 & 6,92 \\ \hline
        101,3 & 700 & 8,31 \\ \hline
    \end{tabular} 
    \caption{Увеличение pH суспензии при добавлении NaOH.}
\end{table}

\begin{figure}[H]
\centering
 \caption{Кривая титрования суспензии раствором гидрооксида натрия. }
\begin{tikzpicture}
\begin{axis} [
    xlabel = {V, \textit{мкл}},
    ylabel = {pH},
    width = 15 cm,
    height = 15 cm,
    grid = major,
    ymin = 0,
    xmin = 0,
    xmax = 900,
    ymax = 10
]
\addplot table [x = a, y = b]{
	a     b    
	0     3.51  
        50    3.53  
        100   3.61 
        150   3.68 
        200   3.78
        250   3.91 
        300   4.06 
        350   4.22 
        400   4.44
        450   4.77 
        500   5.13
        550   5.56 
        600   5.83 
        650   6.92 
        700   8.31    
};
\end{axis}
\end{tikzpicture}
\end{figure}

По кривой титрования определим ёмкость катионита в единицах мг-экв/г:
\begin{equation*}
    \textit{Г}_0^{(2)} = 0,43 \; \frac{ммоль.}{г.}
\end{equation*}

\subsection{Построение изотерм ионного обмена.}

На основании проведенных опытов средняя ёмкость катионита равна 

\begin{equation*}
      \textit{Г}_0  = \frac{\textit{Г}_0^{(1)} + \textit{Г}_0^{(1)}}{2} = (0,418 \pm 0,013) \; \frac{ммоль}{г.}, \quad \varepsilon = 6 \%.
\end{equation*}


Используя полученное значение для ёмкости катионита построим изотермы ионного обмена для катионов натрия, калия и лития.

\begin{figure}[H]
\centering
 \caption{Изотермы ионнного обмена. Г[$\frac{\textrm{ммоль}}{\textrm{г.}}$] }
\begin{tikzpicture}
\begin{axis} [
    xlabel = {$a_{Me^{+}}$/$a_{H^{+}}$},
    ylabel = {$\textit{Г}_{{Me}^+}$/$\textit{Г}_{H^{+}} \cdot 10^{-1}$} ,
    width = 15 cm,
    height = 15 cm,
    grid = major,
    ymin = 0,
    xmin = 0,
]
\legend{ 
	$Na$, 
	$K$, 
	$Li$
};
\addplot table [x = a, y = b]{
	a     b    
	0.04045114 0.15609673 
        0.58582439 0.33718556 
        0.86579072 0.47520577 
        1.14970127 0.59485109 
        1.57602586 0.62355594 
        1.98136488 0.65403431
        2.36646403 0.68644655 
};
\addplot table [x = a, y = b]{
	a     b    
	0.5331076  0.01774323
        6.87302365  0.04024875
        8.25892828   0.05735992
        8.96619008 0.07721284 
        10.31258842 0.08777741
        11.06191105  0.1027666
        12.40761306   0.10943793
        13.60595214  0.11651425
        15.03249472 0.1202536 
};
\addplot table [x = a, y = b]{
	a     b    
	0.40062378  0.02836417 
 6.24990396  0.04544856 
 8.10739689  0.05820717
 9.98169933  0.06666764
 11.16975081  0.07861952
 12.8948333  0.08391953
 14.43781797   0.08952823 
 15.42902126   0.09853314
 16.61829527    0.10500804
 17.66540452   0.1118738 
 18.58082197    0.11915879
};
\end{axis}
\end{tikzpicture}
\end{figure}

Чтобы построить изотермы для Г[$\frac{\textrm{мг.-экв.}}{\textrm{г.}}$] домножим значения по оси абсцисс на молярную массу каждого из металлов (молярная масса и эквивалентная масса совпадают):

\begin{equation*}
    M_{Na} = 23 \frac{\textrm{г.}}{\textrm{моль}}, \quad M_{K} = 39 \frac{\textrm{г.}}{\textrm{моль}}, \quad M_{Li} = 7 \frac{\textrm{г.}}{\textrm{моль}}. 
\end{equation*}



\begin{figure}[H]
\centering
 \caption{Изотермы ионнного обмена. Г[$\frac{\textrm{мг.-экв.}}{\textrm{г.}}$] }
\begin{tikzpicture}
\begin{axis} [
    xlabel = {$a_{Me^{+}}$/$a_{H^{+}}$},
    ylabel = {$\textit{Г}_{{Me}^+}$/$\textit{Г}_{H^{+}}$},
    width = 15 cm,
    height = 15 cm,
    grid = major,
    ymin = 0,
    xmin = 0,
    xmax = 20,
    ymax = 17
]
\legend{ 
	$Na$, 
	$K$, 
	$Li$
};
\addplot table [x = a, y = b]{
	a     b    
	0.04045114  3.59022487 
        0.58582439  7.75526783
        0.86579072 10.92973267 
        1.14970127  13.68157499 
        1.57602586 14.34178665
        1.98136488 15.04278918
        2.36646403   15.78827058     
};
\addplot table [x = a, y = b]{
	a     b    
	0.5331076  0.69198596
        6.87302365  1.56970139
        8.25892828    2.23703676
        8.96619008 3.01130087 
        10.31258842 3.42331901
        11.06191105  4.00789727
        12.40761306    4.26807931
        13.60595214 4.54405578
        15.03249472 4.68989039  
};
\addplot table [x = a, y = b]{
	a     b    
	0.40062378  0.19854917 
 6.24990396  0.31813995 
 8.10739689  0.4074502 
 9.98169933  0.46667348 
 11.16975081  0.55033663
 12.8948333  0.58743674
 14.43781797   0.62669759
 15.42902126   0.68973199
 16.61829527   0.73505625
 17.66540452   0.78311661 
 18.58082197    0.83411154
};
\end{axis}
\end{tikzpicture}
\label{pic:1}
\end{figure}

Определим константы равновесия этих реакций по наклонам прямых, посчитанных с помощью МНК:

\begin{equation*}
    K_{Na^{+}} = (0,22 \pm 0,4), \quad \varepsilon_{Na^{+}} = 18 \%.
\end{equation*}
\begin{equation*}
    K_{K^{+}} = (0,0051 \pm 0,0001), \quad \varepsilon_{K^{+}} = 16 \%.
\end{equation*}
\begin{equation*}
    K_{Li^{+}} = (0,0035 \pm 0,002), \quad \varepsilon_{Li^{+}} = 4 \%.
\end{equation*}

\section{Вывод}
\begin{itemize}
    \item Нами были получены изотермы для процессов ионного обмена в катионите с участием
ионов натрия, калия и лития, были оценены значения констант равновесия. Использованный нами катионит (полисурьмяная кислота SbSiP) проявляет наибольшее сродство к катионам натрия, что можно объяснить более подходящими размерами ионов металла.

Это можно объяснить следующим образом: чем больше заряд обмениваемого иона, тем лучше ионит обменивается им, а если заряды одинаковы, лучше обмениваются ионы, радиус которых больше. Одновременно с этим необходимо учитывать реакционную способность вещества. 

Заряд исследуемых металлов катионов были одинаковы, так что надо учитывать только реакционную способность и радиус атома. Литий наименее реакционно способен, а калий наиболее реакционоспособен, но ионы калия слишком большие, им сложно проникнуть в структуру катионита и обменяться с ними ионами.

Натрий играет роль "золотой середины": небольшой радиус позволяет ему проникать в катионит, а реакционная способность его намного сильнее сем у калия.

\item Изотермы ионного обмена мы строили в предположении, что температура среды не изменялась (иначе это были бы уже не изотермы). Это условие было нарушено в ходе эксперимента в силу невозможности контролировать температуру внешней среды и в силу того, что автору отчета стало душно и он по своей неосторожности во время эксперимента открыл окно.

\item Ошибка при линейной апроксимации прямых на графике \ref{pic:1} получилось достаточно большой. Это может быть связано с тем, что когда ионит насыщен, то ему сложнее обмениваться с раствором ионами, нежели когда он ещё не вступил в ионно-обменную реакцию.

\item Стоит так же учитывать, что при подсчете активностей мы не учитывали коэффициент активности и считали его равным единице.

\end{itemize}
\end{document}
